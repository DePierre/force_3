\documentclass{beamer}
\usepackage[utf8]{inputenc}
\usepackage[french]{babel}
\usepackage{graphicx}
\usepackage{multicol}

\usetheme{Luebeck}
%\usecolortheme{beetle}
% \usecolortheme{beaver}
% \usecolortheme{albatross}
% \usecolortheme{crane}
% \usecolortheme{fly}
% \usecolortheme{seagull}
% \usecolortheme{horse}
% \usecolortheme{whale}


\title{IA41 - Force 3}
\author{Tao Sauvage \and Julien Voisin \and Robin Faury}
\institute[UTBM]{Université de Technologie de Belfort Montbéliard}

\newcommand{\link}[2] {
\item[#1] \scriptsize #2 \normalsize
}


\AtBeginSection[]
{
  \begin{frame}<beamer>
    \frametitle{Plan}
    \tableofcontents[currentsection,currentsubsection]
  \end{frame}
}

\begin{document}

\begin{frame}
	\titlepage
\end{frame}


\begin{frame}{Plan}
    \tableofcontents
\end{frame}


\section{Analyse}

\begin{frame}{Un jeu de société}
  \begin{figure}
    \includegraphics[height=0.6\textheight]{./pix/plateau}
    \centering
    \caption{Boîte de teeko}
  \end{figure}
\end{frame}


\begin{frame}{Modélisation de la partie}
  \begin{figure}
    \includegraphics[width=0.9\textwidth]{./pix/arbre}
    \centering
    \caption{Un arbre représentant une partie}
  \end{figure}
\end{frame}


\begin{frame}{Évaluation d'un plateau}
    \begin{block}{Fonction d'évaluation}
        $$-100 \leq evaluation(plateau) \leq 100$$
    \end{block}
    \begin{block}{Pondération en fonction du nombre de pions alignés:}
        \begin{description}
            \item[1 pion seul:] 1 point 
            \item[2 pions alignés:] 5 point
            \item[3 pions alignés:] 100 points
        \end{description}
    \end{block}
\end{frame}

\begin{frame}{Évaluation d'un plateau : Exemple}
    \begin{columns}
        \begin{column}[l]{0.5\textwidth}
            \begin{figure}
                \includegraphics[width=1.1\textwidth]{./pix/evaluation}
                \centering
                \caption{Evaluation d'un plateau}
            \end{figure}
        \end{column}
        
        \begin{column}[r]{0.5\textwidth}
            \begin{description}
                \item[Joueur Rouge] $1 + 1 + 1 = 3$
                \item[Joueur Bleu] $5 + 1 + 1 + 1 = 8$
                \item[Évaluation pour Bleu] 8 -3 = 5
            \end{description}
        \end{column}

  \end{columns}
\end{frame}

\begin{frame}{Méthode de résolution}
  \begin{figure}
    \includegraphics[height=0.7\textheight]{./pix/methode}
    \centering
    \caption{Méthode de résolution}
  \end{figure}
\end{frame}


\begin{frame}{Algorithme utilisé}
  %\begin{figure}
    %FIXME
    %\includegraphics[height=0.7\textheight]{pics/alphabeta}
    %\centering
    %\caption{variante négamax du minimax avec élagage $\alpha\beta$}
  %\end{figure}
\end{frame}

\section{Choix d’implémentation}
%FIXME
    \begin{frame}{Modules}
        \begin{description}
            \item[mod\_regles :] règles du jeu, conditions de victoire.
            \item[mod\_jeu :] prédicat de jeu
            \item[mod\_eval :] évaluation d'un plateau, algo négamax
            \item[mod\_ui :] Interface utilisateur
            \item[force3 :] Fichier principal
        \end{description}
    \end{frame}

\begin{frame}{Représentation des données}
    \begin{block}{Coup}
          Liste de la forme \texttt{[Case départ, Case arrivée, Id]}
    \end{block}
    \begin{block}{Plateau}
        Liste de neuf entiers $\in \left\{-1, 0, 1, 2\right\}$, avec:
        \begin{description}
	        \item [-1] : case vide
	        \item [0] : case sans pion
	        \item [1] : pion du joueur 1
	        \item [2] : pion du joueur 2
        \end{description}
        Exemple : \texttt{[1, 1, 2, 0, -1, 0, 1, 0, 0]}
    \end{block}
\end{frame}

%     \begin{frame}{Modules}
%         \begin{description}
%             \item[coup(+JR, +PL, ?Coups)] Trouver les coups possibles.
%             \item[mod\_jeu :] prédicat de jeu
%             \item[mod\_eval :] évaluation d'un plateau, algo négamax
%             \item[mod\_ui :] Interface utilisateur
%             \item[force3 :] Fichier principal
%         \end{description}
%     \end{frame}


\section{Résultats}

\begin{frame}{Utilité de l'élagage $\alpha\beta$}
Sans élagage, une profondeur de 10 fait exploser les calculs.
\end{frame}


\begin{frame}{Questions}
Questions ?
\end{frame}
\end{document}
